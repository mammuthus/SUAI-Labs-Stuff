\documentclass[a4paper]{article}
\usepackage{soulutf8}
\usepackage[14pt]{extsizes}
\usepackage[utf8]{inputenc}
\usepackage[english, german, latin, russian]{babel}
\usepackage{graphicx}
\usepackage{setspace,amsmath}
\usepackage{blindtext}
\usepackage{listings}
\usepackage{xcolor}
\usepackage{hyperref}

\usepackage[left=20mm, top=15mm, right=15mm, bottom=15mm, nohead, footskip=10mm]{geometry}
\parindent = 0cm
\graphicspath{ {/Users/omavel/My Drive/Магистратура/ИБ/images} } 
\newcommand{\cbox}[2][green]{%
  \colorbox{#1}{\parbox{\dimexpr\linewidth-2\fboxsep}{\strut #2\strut}}%
}
\lstset{extendedchars=\true,keepspaces=true,
basicstyle=\ttfamily,columns=flexible,
escapechar=$,escapebegin=\[,escapeend=\]
    backgroundcolor=\color{backcolour},   
    commentstyle=\color{codegreen},
    keywordstyle=\color{magenta},
    numberstyle=\color{codegray},
    stringstyle=\color{codepurple},
    breakatwhitespace=false,         
    breaklines=true,                 
    captionpos=b,                        
    numbersep=5pt,                  
    showspaces=false,                
    showstringspaces=false,
    showtabs=false,                  
    tabsize=2
}

\begin{document}
\begin{center}
\hfill \break
\small{МИНИСТЕРСТВО ОБРАЗОВАНИЯ И НАУКИ РОССИЙСКОЙ ФЕДЕРАЦИИ}\\
\footnotesize{федеральное государственное автономное образовательное учреждение высшего}\\ 
\footnotesize{образования}\\
\small{«САНКТ-ПЕТЕРБУРГСКИЙ ГОСУДАРСТВЕННЫЙ УНИВЕРСИТЕТ АЭРОКОСМИЧЕСКОГО ПРИБОРОСТРОЕНИЯ»}\\
\hfill \break
\small{\textbf{КАФЕДРА ВЫЧИСЛИТЕЛЬНЫХ СИСТЕМ И СЕТЕЙ}}\\
\hfill \break
\hfill \break
\hfill \break
\hfill \break
\end{center}
\normalsize{ОЦЕНКА\\
ПРЕПОДАВАТЕЛЬ}\\\\
\begin{tabular}{c c c}
\underline{\hspace{0.8cm}доц., к.т.н., доц.\hspace{0.8cm}} & \underline{\hspace{1cm} \phantom{ц baseline hack} } & \underline{\hspace{0.8cm}В.С. Коломойцев\hspace{0.8cm}} \\
\hfill\break
\footnotesize{должность, уч. степень, звание} & \footnotesize{подпись, дата} & \footnotesize{инициалы, фамилия}\\\\
\end{tabular}\\
\hfill \break
\hfill \break
\begin{center}
\normalsize{ОТЧЕТ О ЛАБОРАТОРНОЙ РАБОТЕ №1\\
\hfill \break
Методы и способы работы с редакторами\\
\hfill \break
По дисциплине: Безопасность и защита информации\\
в информационных системах}\\
\hfill \break
\hfill \break
\end{center}
\begin{tabular}{l c c c}
РАБОТУ ВЫПОЛНИЛ\\
СТУДЕНТ ГР. & \ul{  Z0440M  } & \ul{  25.12.2021  } & \underline{\hspace{0.4cm}Д.А. Митрофанов\hspace{0.4cm}}\\
\hfill\break
\end{tabular}\hfill \break
\hfill \break
\hfill \break
\hfill \break
\hfill \break
\hfill \break
\begin{center}Санкт-Петербург 2021 \end{center}
\thispagestyle{empty}
\newpage

\section{Цель работы}
Изучить методы анализа, форматирования и хранения текстовой, графической и иной мультимедийной информации.
\section{Работа с языком \TeX в редакторе \LaTeX}

\subsection{Текст разного кегля и на разных языках}
\subsubsection{Разные кегли}
Основой размер (кегль) шрифта определяется классом документа. 
Кегль текстового фрагмента можно изменить пропорционально, использовав соответствующие теги:
\begin{center}
\renewcommand{\arraystretch}{2}
\begin{tabular}{l l}

\begin{lstlisting}[language=TeX]
\tiny
\end{lstlisting} & \tinyТрехчастная фактурная форма позиционирует контрапункт контрастных фактур\\

\begin{lstlisting}[language=TeX]
\scriptsize
\end{lstlisting} & \scriptsizeСкалярное поле иногда может отображать перекрестный сходящийся ряд\\

\begin{lstlisting}[language=TeX]
\footnotesize
\end{lstlisting} & \footnotesizeМетод последовательных приближений регрессийно искажает форшлаг\\

\begin{lstlisting}[language=TeX]
\small
\end{lstlisting} & \smallБанкротство даёт большую проекцию на оси, чем акционерный суд\\

\begin{lstlisting}[language=TeX]
\normalsize
\end{lstlisting} & \normalsizeИзлучение ментально индуцирует гамма-квант Ориона\\

\begin{lstlisting}[language=TeX]
\large
\end{lstlisting} & \largeСтруктурализм испускает этиловый интеллект\\

\begin{lstlisting}[language=TeX]
\Large
\end{lstlisting} & \LargeПарадигма дает субсветовой аутотренинг\\

\begin{lstlisting}[language=TeX]
\LARGE
\end{lstlisting} & \LARGEСовременная критика иллюзорна\\

\begin{lstlisting}[language=TeX]
\huge
\end{lstlisting} & \hugeОбразование амбивалентно\\

\begin{lstlisting}[language=TeX]
\Huge
\end{lstlisting} & \HugeБхутавада вероятна\\

\hfill\break
\end{tabular}
\end{center}
\subsubsection{Вставки на других языках}
Текстовые вставки на разных языках обеспечиваются подключением соответствующих пакетов помощью 
\begin{lstlisting}[language=TeX]
\usepackage[language]{babel}
\end{lstlisting}
\thispagestyle{empty}
\newpage
При использовании нескольких языков, они задаются через запятую:
\begin{lstlisting}[language=TeX]
\usepackage[english, german, latin, russian]{babel}
\end{lstlisting}\hfill\break
Пакет babel позволяет отображать специальные символы алфавита, а так же автоматически переводит структурные элементы документа. 

Например, ниже был выведен текст-рыба через 
\begin{lstlisting}[language=TeX]
\selectlanguage{german}
\begin{abstract}
\blindtext
\end{abstract}
\end{lstlisting}
\cbox[lightgray!90]{
\selectlanguage{german}
\begin{abstract}
\blindtext
\end{abstract}
}
\selectlanguage{russian} \\
При указании latin в selectlanguage заголовок также локализируется:\\
\cbox[lightgray!90]{
\selectlanguage{latin}
\begin{abstract}
\blindtext
\end{abstract}
}
\selectlanguage{russian}

Некоторые пакеты (как, например, blindtext для генерации рыбы) поддерживают изменение языка.\\
Для возврата к исходному языку: 
\begin{lstlisting}[language=TeX]
\selectlanguage{russian} 
\end{lstlisting}
\thispagestyle{empty}
\newpage
\subsection{Рисунки и таблицы}
\subsubsection{Рисунки}
Для вставки изображений может использоваться пакет graphicx.
Удобно с самого начала задать директорию по умолчанию, где хранятся изображения:
\begin{lstlisting}[language=TeX] 
\graphicspath{{./images/}} 
\end{lstlisting}
и вызывать их по относительным ссылкам через
\begin{lstlisting}[language=TeX] 
\includegraphics{universe}
\end{lstlisting} 
\begin{figure}[h]
    \centering
    \includegraphics[width=\textwidth]{poster-closeup.jpeg}
    \caption{Пример вставки изображения в \LaTeX}
    \label{fig:mesh1}
\end{figure}

Масштаб отображения картинки может быть задан параметром scale, либо динамически, например, в зависимости от ширины текста в документе (как в приведенном примере на Рисунке \ref{fig:mesh1}): \begin{lstlisting}[language=TeX] 
\includegraphics[width=\textwidth]{poster-closeup.jpeg}
\end{lstlisting}
\thispagestyle{empty}
\newpage
\subsubsection{Таблицы}
\LaTeX позволяет создавать сложные таблицы с различными параметрами выравнивания текста внутри ячеек, с настройкой ширины столбцов и разделителей.
\hfill\break
\\Параметры столбца: c — выравнивание содержимого по центру, r — по правому краю, l — по левому, p — столбец заданной ширины, | — разделители.
Чтобы прочертить линии между строками используют команду hline. Для того, чтобы объединить ячейки нескольких столбцов используют multicolumn. \\
\hfill\break
\begin{tabular}{ |p{4cm}|p{4cm}|p{4cm}|  }
\hline
\multicolumn{3}{|c|}{Country List} \\
\hline
Country Name & ISO ALPHA 2 & ISO ALPHA 3 \\
\hline
Afghanistan & AF & AFG \\
Aland Islands & AX & ALA \\
Albania & AL & ALB \\
\hline
\end{tabular}
\hfill\break
\\Исходный код:
\begin{lstlisting}[language=TeX] 
\begin{tabular}{ |p{4cm}|p{4cm}|p{4cm}|  }
\hline
\multicolumn{3}{|c|}{Country List} \\
\hline
Country Name & ISO ALPHA 2 & ISO ALPHA 3 \\
\hline
Afghanistan & AF & AFG \\
Aland Islands & AX & ALA \\
Albania & AL & ALB \\
\hline
\end{tabular}
\end{lstlisting}
\thispagestyle{empty}
\newpage
\subsection{Библиографическое описание и ссылки}

Для демонстрации работы библиографического описания в \LaTeX, приведу фрагмент из статьи \emph{Early assessment of the clinical severity of the SARS-CoV-2 Omicron variant in South Africa doi: https://doi.org/10.1101/2021.12.21.21268116}:
\\
\\
\cbox[lightgray!90]{On 24 November 2021, the Network for Genomics Surveillance of South Africa (NGS-SA) reported a new variant of SARS-CoV-2 which had been detected from specimens collected on 14 November 2021 in South Africa, originally assigned to the lineage B.1.1.529 \cite{NatInstitute}. The WHO, on the recommendation of the Technical Advisory Group on SARS-CoV-2 Virus Evolution, designated B.1.1.529 as Omicron \cite{WHO}, the fifth VOC, on 26th November 2021}
\\\\
Список ссылок необходимо составить заранее, задав для каждого референса уникальный индекс. В данном случае использовался следующий код:
\begin{lstlisting}[language=TeX]
\begin{thebibliography}{5}
\bibitem{NatInstitute}
National Institute for Communicable Diseases 
\emph{COVID19 Wkly. Epidemiol. BRIEF, WEEK 48 2021} 
https://www.nicd.ac.za/wpcontent/uploads/2021/12/COVID-
19-Weekly-Epidemiology-Brief-week-48-2021.pdf 
(accessed Dec 9, 2021)
\bibitem{WHO}
World Health Organization. \emph{Classification of Omicron 
(B.1.1.529): SARS-CoV-2 Variant of
Concern} https://www.who.int/news/item/26-11-2021-
classification-of-omicron-(b.1.1.529)-
sars-cov-2-variant-of-concern (accessed Dec 10, 2021)
\end{thebibliography}
\end{lstlisting}

Ссылки вставляются в текст через директиву cite с указанием индекса:
\begin{lstlisting}[language=TeX]
\cite{NatInstitute}
\end{lstlisting}
\thispagestyle{empty}
\newpage
Сгенерированный библиографический список выглядит так:\\\\
\selectlanguage{english} 
\cbox[lightgray!90]{
\begin{thebibliography}{5}
\bibitem{NatInstitute}
National Institute for Communicable Diseases \emph{COVID19 Wkly. Epidemiol. BRIEF, WEEK 48 2021} https://www.nicd.ac.za/wpcontent/uploads/2021/12/COVID-19-Weekly-Epidemiology-Brief-week-48-2021.pdf (accessed Dec 9, 2021)
\bibitem{WHO}
World Health Organization. \emph{Classification of Omicron (B.1.1.529): SARS-CoV-2 Variant of
Concern} https://www.who.int/news/item/26-11-2021-classification-of-omicron-(b.1.1.529)-
sars-cov-2-variant-of-concern (accessed Dec 10, 2021)
\end{thebibliography}
}
\selectlanguage{russian} 

На русском языке ссылки на литературу встраиваются абсолютно так же.\\
Пример подобной ссылки:\\\\
\cbox[lightgray!90]{Особенностью метода Фибоначчи является возможность экономить
дорогостоящие вычисления целевой функции f(x). Действительно, начиная со
второго шага, значение функции в точке c известно с предыдущего шага, поэтому
остаётся вычислить только значение в отраженной точке x\cite{Opt}}
\hfill\break
И результат работы thebibliography:\\\\
\cbox[lightgray!90]{
\begin{thebibliography}{5}
\bibitem{Opt}
Оптимальное управление, Галеев Э.М., Зеликин М.И., Конягин С.В. и др.\emph{Московский Центр Непрерывного Математического Образования
 (МЦНМО), 2008.}
 \end{thebibliography}
}
\thispagestyle{empty}
\newpage
\subsection{Сноски}
Для введения сносок применяется директива footnote, которая может принимать один или два параметра: текст сноски и индекс.\\\\
Для демонстрации я воспроизвел фрагмент статьи \emph{\href{https://artguide.com/posts/1251}{Конверсия ошибки: глитч-арт в компьютерных играх (Алина Латыпова)}}. Фрагмент умышленно взят из середины статьи, поэтому нумерация сносок начинается не с единицы. \\Текст внутри сносок воспроизведен "как есть".\\

\emph{Если обратиться к классификации английского исследователя цифрового искусства Имана Моради\footnote[16]{Обращение к предложенной классификации обусловлено не столько ее оригинальностью, сколько систематичностью. Исследователем рассмотрены и распределены по группам наиболее часто встречающиеся типы проектов в стиле глитч-арт. Подробнее см.: Moradi I. Seeking Perfect Imperfection. A personal retrospective on Glitch Art // Vector. Vol. 6. Errors and glitches. July 2008. URL: http://virose.pt/vector/x\_06/moradi.html (дата обращения: 17.06.2016).}, глитч-арт можно разделить на «чистый» и «глитч-подобный». Или: естественный\footnote[17]{«Естественность» в данном случае означает сообразность с игровой средой. То, что логично вписывается в цифровую среду, является естественным для нее.} и рукотворный. Естественный глитч-арт представляет собой сбои в коде, изображениях и другие баги, возникающие в ходе исполнения программы, которые заметил и зафиксировал художник. В то время как рукотворный глитч-арт — это намеренное искажение (модификация) медиаконтента\footnote[18]{Существует множество обучающих материалов по глитч-арту, а также специальные онлайн-сервисы, предлагающие преобразовать загружаемые данные в файл в стиле глитч-арта. В качестве примера можно назвать сервис Image Glitch Tool. URL: https://snorpey.github.io/jpg-glitch (дата обращения: 17.06.2016).} }\\\\
\hfill\break
Код (содержимое сносок сокращено):\\
\begin{lstlisting}[language=TeX]
 Если обратиться к классификации английского исследователя цифрового 
 искусства Имана Моради\footnote[16]{...}, глитчарт можно 
 разделить на «чистый» и «глитч подобный». Или: 
 естественный\footnote[17]{...} и рукотворный. Естественный глитчарт 
 представляет собой сбои в коде, изображениях и другие баги, 
 возникающие в ходе исполнения программы, которые заметил 
 и зафиксировал художник. В то время как рукотворный глитч арт  
 это намеренное искажение (модификация) медиаконтента\footnote[18]{...}
\end{lstlisting}
\thispagestyle{empty}
\newpage
\section{Выводы}
Были изучены методы анализа, форматирования и хранения текстовой, графической и иной информации, кодирование разметки в нотации \TeX с использованием редактора \TeX Shop. Отчет был сверстан в \TeX, исходный код доступен \href{https://artguide.com/posts/1251}{по ссылке}
\end{document}